\documentclass[hidelinks__VERSION__]{adamyi-cv}
\usepackage{fontspec}
\newfontfamily{\hei}{STHeiti-Light}
\newfontfamily{\heir}{STHeiti}
%\usepackage[UTF8]{ctex}
\DeclareSymbolFont{extraup}{U}{zavm}{m}{n}
\DeclareMathSymbol{\varheart}{\mathalpha}{extraup}{86}
\begin{document}

\revision{__REVISION__}
\footertext{This version was compiled on \today\ at \DTMcurrenttime. Source code available at \href{https://github.com/hackroid/cv-new}{https://github.com/hackroid/cv-new} under MIT license.}

\header{\heir 赵}{ \hei 博弘}{\hei 本科 | 南方科技大学}

%----------------------------------------------------------------------------------------
%	SIDEBAR SECTION
%----------------------------------------------------------------------------------------

\begin{aside} % In the aside, each new line forces a line break
\section{\heir 联络方式}
(+86)17620412037
{\hei 微信\textrm{: w1231824}}
{\hei QQ\textrm{: 1290756262}}
~
\href{mailto:ooozbh@gmail.com}{ooozbh@gmail.com}
\href{https://blog.hackroid.com}{blog.hackroid.com}
\href{https://github.com/hackroid}{GitHub: hackroid}
~
\href{https://www.linkedin.com/in/%E5%8D%9A%E5%BC%98-%E8%B5%B5-b7ab09136/}{\hei 领英\textrm{: Bohong-Zhao}}
\href{https://www.facebook.com/Hackro1d}{\hei 脸书\textrm{: Hackroid}}
\section{\heir 语言能力}
{\hei 中文 \textbf{\hei 母语}
\hei 英语 \textbf{\hei 流畅}
\hei 日语 \textbf{\hei 初学}}
\section{\heir 编程}
\textbf{\hei 熟练}
{\color{red} $\varheart$} Python
Java, C++
\textbf{\hei 了解}
CSS3 \& HTML5
Kotlin, Dart
SQL, PHP, JavaSciprt
Bash/Shell, Spim
MATLAB, \LaTeX
\section{IDE\&editor}
{\color{red} $\varheart$} PyCharm
PHPStorm
~
{\color{red} $\varheart$} VSCode
Vim
Sublime
\versionsection
\end{aside}

%----------------------------------------------------------------------------------------
%	EDUCATION SECTION
%----------------------------------------------------------------------------------------

\section{\heir 教育\heir 经历}

\begin{entrylist}

%------------------------------------------------

\entry
{\href{https://www.ubc.ca/}{\heir 加拿大英属哥伦比亚大学(UBC)}, \hei 加拿大BC温哥华}
{2018.07 -- 2018.08}
{\hei \emph{暑期项目}\\
主要学习算法设计与JavaScript网络应用开发
}

\entry
{\href{https://www.sustech.edu.cn/}{\heir 南方科技大学(SUSTech)}, \hei 中国广东深圳}
{2016.09 -- 2020.07}
{\hei \emph{\textbf{\hei 本科} 计算机科学与工程}\\
相关课程:数据结构与算法分析,离散数学,数字逻辑,计算机系统设计与应用,计算机组成原理,\\
数据库原理,概率论与数理统计,人工智能,计算机网络,嵌入式系统与微机原理,面向对象分析与\\
设计,软件工程,C/C++程序设计,计算机视觉,认知科学导论,操作系统(在学),机器人(在学)
}

\end{entrylist}

%----------------------------------------------------------------------------------------
%	RESEARCH SECTION
%----------------------------------------------------------------------------------------

\section{\heir 实习\heir 经历}

\begin{entrylist}

%------------------------------------------------

\entry
{\href{http://www.doodod.com/}{\heir 独到科技(北京)有限公司}}
{2017.08}
{\emph{\hei 兼职开发助理}
\begin{itemize}
\item {\hei 在独到科技工作并协助开发了公司项目使用的用户画像标签系统}
\end{itemize}}

%------------------------------------------------

\end{entrylist}

%----------------------------------------------------------------------------------------
%	AWARDS SECTION
%----------------------------------------------------------------------------------------

\section{\heir 获奖\heir 经历}

\begin{entrylist}

%------------------------------------------------

\entry
{\heir 优秀大学长}
{2017.06 -- 2018.06}
{\hei \emph{\textbf{\hei 南方科技大学}}\\
作为在校书院学生助理,协助辅导员处理书院事务,帮助新生开展活动融入校园生活。
}

%------------------------------------------------

\entry
{\heir 本科生奖学金}
{2016.09}
{\hei \emph{\textbf{\hei 南方科技大学}}\\
对入学时优秀的新生进行嘉奖以资鼓励
}

%------------------------------------------------

\end{entrylist}

%----------------------------------------------------------------------------------------
%   PROJECTS SECTION
%----------------------------------------------------------------------------------------

\section{\heir 科研\heir 课题}

\begin{entrylist}

%------------------------------------------------

\entry
{\href{https://github.com/hackroid/pMOEA-D}{\heir MOEA/D 算法的并行化实现及其在特征选择上的应用}}
{2018.09 -- 2019.06}
{\emph{\hei 已完成}
\begin{itemize}
    \item \hei 语言:Python
\end{itemize}
{\hei
在\href{http://cse.sustech.edu.cn/en/people/view/people_id/55/sort_id/9/pid/}{\hei 石渕久生}教授指导下的本科生团队科研项目。特征选择在数据挖掘与机器学习中占有举足轻重的地\\
位,而近年来演化计算在特征选择问题上取得了显著的进展。本项目旨在为 MOEA/D 算法提供并行化\\
环境,使用孤岛模型,探究发现对于区块间高效的交流方式以提高整体上获取帕累托最优解集的速度\\
,并用特征选择问题来验证我们的想法。
}}

%------------------------------------------------


\end{entrylist}

\pagebreak


%----------------------------------------------------------------------------------------
%	PROJECTS SECTION
%----------------------------------------------------------------------------------------

\section{\heir 项目\heir 经验}

\begin{entrylist}

%------------------------------------------------

\entry
{\heir Snappat -- 基于解谜的新型社交方式}
{2019.02 -- 2019.06}
{\emph{\hei 已完成}
\begin{itemize}
    \item \hei 语言:Java
    \item \hei 框架:安卓原生
\end{itemize}
{\hei
这是一个软件工程课程的课内项目,我与其他五名组员一同开发一个社交手机应用,其囊括解谜游戏\\
、图像识别与机器学习等诸多特性。此项目更注重于学习安卓程序的协作开发流程与测试。
}}

%------------------------------------------------

\entry
{\href{https://github.com/hackroid/MqttSensorDemo}{\heir 传感器数据实时传输应用}}
{2018.09 -- 2019.01}
{\emph{\hei 已完成}
\begin{itemize}
    \item \hei 语言:Java
    \item \hei 框架:安卓原生
    \item \hei 负责:全栈开发与服务器维护
\end{itemize}
{\hei
基于 MQTT 协议搭建一个安卓应用使其能够在多终端之间互相传送传感器实时数据。MQTT 是一个轻\\
量级的消息推送协议,尤其在于其便于使用并对延迟进行了优化。我在这个计算机网络课程项目中主\\
要学习了 MQTT 协议的通信原理。
}}

%------------------------------------------------

\entry
{\href{https://github.com/zhaoweizhong/Faculty-Reservation}{\heir 导师预约系统}}
{2018.09 -- 2019.01}
{\emph{\hei 已完成}
\begin{itemize}
    \item \hei 语言:PHP
    \item \hei 框架:Laravel
    \item \hei 负责:后端PHP爬虫,从 researchgate 上爬取研究人员最近发表文章数据
\end{itemize}
{\hei
在一个五人小组内共同开发一个系统以帮助南方科技大学的学生和教职工管理他们的预约。此系统可\\
以清晰地向用户呈现预约对象的可利用时间,高效地规划管理自己的预约事件,极大减少了通信交流\\
过程中可能产生的时间消耗与错误。本系统包括一些常用功能比如用户注册登录信息修改,预约管理\\
,近期项目搜索等。
}}


%------------------------------------------------

\entry
{\heir 地震可视化软件}
{2017.02 -- 2017.06}
{\emph{\hei 已完成}
\begin{itemize}
    \item \hei 语言:Java
    \item \hei 平台:JavaFX
    \item \hei 负责:后端Java爬虫,从地震发布网站爬取地震数据
\end{itemize}
{\hei
使用 Java GUI 对地震统计数据在地图和图表中进行可视化呈现。在这次项目中,除了学习 Java 语言,\\
我还第一次独自学习了网络请求的发送与处理和 HTML 文档的分析与切片。这是 Java 编程2(高级)课\\
程的三人组队项目。
}}

%------------------------------------------------

\entry
{\heir 航班预约系统}
{2016.09 -- 2017.01}
{\emph{\hei 已完成}
\begin{itemize}
    \item \hei 语言:Java, HTML\&CSS, SQL
    \item \hei 实现:JSP
    \item \hei 负责:全栈开发
\end{itemize}
{\hei
领导一个三人小组一起开发一个航班预约系统,系统包括一些常用必要功能,比如用户注册,个人信息\\
修改,航班搜索、预约、修改,管理员界面等。 此项目本是 Java 编程1(初级)课程的项目,而课程上\\
没有涉及用户界面的知识,所以自学并使用了 JSP 这个可以运用 Java 知识以及 HTML 界面设计的技术\\
,同时也使用到了 Material Design(MD) 这个优秀而美观的设计语言,后端使用 MySQL 实现简易的数\\
据增查删改。
}}

%------------------------------------------------

\end{entrylist}

\end{document}
